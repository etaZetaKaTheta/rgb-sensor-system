\documentclass[10pt,a4paper]{report}
\usepackage[utf8]{inputenc}
\usepackage[T1]{fontenc}
\usepackage[ngerman]{babel}
\usepackage{amsmath}
\usepackage{amsfonts}
\usepackage{amssymb}
\usepackage{makeidx}
\usepackage{graphicx}
\usepackage{fancyhdr}
\usepackage{color}
\usepackage{listings}
\usepackage{hyperref}
\title{Dokumentation RGB-Sensor}
\author{Julius Hahl, Maximilian Trautwein und Sebastian Köhler, 11BG1}
\pagestyle{fancy}

\lstloadlanguages{bash, Python}
\lstset{alsolanguage=bash}
\lstset{alsolanguage=Python}

\lhead{\slshape \rightmark}
\chead{}
\rhead{\slshape \leftmark}

\lfoot{}
\cfoot{\thepage}
\rfoot{}

\renewcommand{\headrulewidth}{0.4pt}
\renewcommand{\footrulewidth}{0pt}

\lstset{language=Python,
	basicstyle=\scriptsize\ttfamily,
	keywordstyle=\color{red}\bfseries,
	identifierstyle=\color{blue},
	commentstyle=\color{DarkGreen},
	stringstyle=\ttfamily,
	breaklines=true,
	numbers=left,
	numberstyle=\tiny,
	frame=single,
	backgroundcolor=\color{myGrey},
	caption={Python-Code},
	tabsize=2}
\definecolor{myGrey}{gray}{0.9}
\definecolor{DarkGreen}{rgb}{0.2,0.5,0.6}





\begin{document}
	\maketitle
	\newpage
	\tableofcontents
	\newpage
	
	
	\part{Dokumentation}
	\chapter{Color-Detection}
	\newpage
	\section{Programm starten}
	\newpage
	\section{Farbe Auslesen}
	\newpage
	
	
	
	\part{Code}
	\chapter{Quellcode}
	
	
	\newpage
	\section{Main.py}
	\begin{lstlisting}
		#!/usr/bin/python3
		
		import cv2 as cv
		import socket
		
		from null_preview import *
		from picamera2 import *
		
		currentColor = ""
		lastColor = ""
		sock = socket.socket(socket.AF_INET, socket.SOCK_STREAM)
		server_address = ('192.168.137.123', 18769)
		print('starting up on {} port {}'.format(*server_address))
		sock.bind(server_address)
		picam2 = Picamera2()
		preview = NullPreview(picam2)
		picam2.configure(picam2.preview_configuration(main={"size":(640, 480)}))
		picam2.start()
		sock.listen(100)
		connection, client_address = sock.accept()
		
		def evaluate_current_frame():
		img = picam2.capture_array()
		
		img = cv.cvtColor(img, cv.COLOR_BGR2RGB)
		
		Z = img.reshape((-1,3))
		# convert to np.float32
		Z = np.float32(Z)
		# define criteria, number of clusters(K) and apply kmeans()
		criteria = (cv.TERM_CRITERIA_EPS + cv.TERM_CRITERIA_MAX_ITER, 10, 1.0)
		K = 1
		ret,label,center=cv.kmeans(Z,K,None,criteria,10,cv.KMEANS_RANDOM_CENTERS)
		# Now convert back into uint8, and make original image
		center = np.uint8(center)
		res = center[label.flatten()]
		res2 = res.reshape((img.shape))
		
		
		
		(b, g, r) = cv.split(res2)
		
		b_mean = np.mean(b)
		g_mean = np.mean(g)
		r_mean = np.mean(r)
		
		# displaying the most prominent color
		if (b_mean > g_mean and b_mean > r_mean):
		currentColor = "blue"
		elif (g_mean > r_mean and g_mean > b_mean):
		currentColor = "green"
		else:
		currentColor = "red"
		
		message = currentColor.encode()
		connection.sendall(message)
		
		def close_socket():
		sock.close()
		
		while True:
		try:
		data = connection.recv(16)
		dataBuffer = data.decode('utf-8')
		if(dataBuffer == "getcolor"):
		evaluate_current_frame()
		elif(dataBuffer == "closesocket"):
		close_socket()
		
		
		except OSError:
		print("STOPPED")
		sock.close()
		break
		
		except KeyboardInterrupt:
		print("STOPPED")
		sock.close()
		break
	\end{lstlisting}
	\newpage
	
	\section{client.py}
	\begin{lstlisting}
		# socket_echo_client.py
		import re
		import socket
		import sys
		import tkinter as tk
		
		sock = socket.socket(socket.AF_INET, socket.SOCK_STREAM)
		
		getColorFuncName = "getcolor"
		sockCloseFuncName = "closesocket"
		
		def connect_to_server():
		# connect socket to port where server is listening
		server_address = ('192.168.137.123', 18769)
		print('connecting to {} port {}'.format(*server_address))
		sock.connect(server_address)
		
		def request_color():
		message = getColorFuncName.encode()
		sock.sendall(message)
		
		data = sock.recv(16)
		print(data.decode('utf-8'))
		
		def close_socket():
		message = sockCloseFuncName.encode()
		sock.sendall(message)
		
		root = tk.Tk()
		root.geometry("150x100")
		
		label1 = tk.Label(root, text="RGB-Sensor-System")
		label1.pack()
		
		schaltf1 = tk.Button(root, text="Connect to server", command=connect_to_server)
		schaltf1.pack()
		
		schaltf2 = tk.Button(root, text="Request color", command=request_color)
		schaltf2.pack()
		
		root.mainloop()
		
	\end{lstlisting}
\end{document}